\section{Discussion} \label{sec:discussion}

% Strength: IMF vs SVD and JPEG
All our comparative results (Figure \ref{fig:rate_distortion} and Figure \ref{fig:imagenet_classification}) consistently show that our IMF method outperforms JPEG in both maintaining image quality and preserving visual semantics at low bit rates and remains comparable at higher bit rates. Moreover, IMF consistently demonstrates superior performance compared to SVD across all bit rates. This superiority can be attributed to the quantization-free nature of IMF, which allows for more accurate reconstruction. In contrast, the high sensitivity of SVD to quantization errors during encoding and decoding degrades reconstruction quality.

% Strength: IMF visual quality vs visual semantic
\remove{Although the rate-distortion curves (Figure \ref{fig:rate_distortion}) show that IMF is surpassed by JPEG at a bit rate of around 0.25 bpp, the ImageNet results (Figure \ref{fig:imagenet_classification}) reveal a take-over point at around 0.3 bpp. This small shift suggests that the superiority of IMF over JPEG at low bit rates in terms of preserving visual semantics is even greater than its advantage in maintaining image quality.}

% Limitation 1
As observed in Figure \ref{fig:bounds_ablation_psnr}, contracting the IMF factor bounds from $[-128, 127]$ to $[-16, 15]$ consistently improves the rate-distortion performance. Generally, narrowing the factor bounds $[\alpha, \beta]$ can potentially lower the entropy, thereby improving the effectiveness of lossless compression in the final stage of our framework and subsequently reducing the bit rate. However, this reduction in entropy comes at the cost of increased reconstruction error, as the feasible set in \eqref{eq:imf_problem} becomes more constrained. This trade-off between entropy and reconstruction quality limits the compression performance of IMF. Therefore, it would be beneficial\remove{to relax the factor bounds to some extent}\remove{to contract the factor bounds less tight} to moderately expand the factor bounds $[\alpha, \beta]$ while simultaneously controlling the entropy of the elements in the factor matrices. We plan to address this in the future by incorporating an entropy-aware regularization term into the current IMF objective function.

% Limitation 2
Patchification with an appropriate patch size (e.g., $(8, 8)$) helps capture local spatial dependencies and, as confirmed by our results in Figure \ref{fig:patchsize_ablation_psnr}, positively impacts the performance of IMF and SVD. However, discontinuities at patch boundaries can introduce \emph{blocking artifacts}, similar to JPEG compression at very low bit rates (see the building image example in Figure \ref{fig:qualitative_comparison}). Moreover, while JPEG suffers more from \emph{color distortion} (e.g., \emph{color bleeding} and \emph{color banding}) at low bit rates, IMF and SVD are more affected by \emph{blurriness}, as observed in the seascape image example in Figure \ref{fig:qualitative_comparison}. As a potential solution for future work, a deep neural network could be trained to remove these artifacts and then integrated as a post-processing module to further enhance the quality of IMF-compressed images.

